

%%%%%%%%%%%%%%%%%%%%%%%%%%%%%%%%%
%%%%%  PREAMBULE/PREAMBLE %%%%%%%
%%%%%%%%%%%%%%%%%%%%%%%%%%%%%%%%%

%%%Layout%%%  Fonte / Police / Orthographe / Guillemets /Accentualtion

\documentclass[a4paper ,12pt]{book}

\usepackage[utf8]{inputenc} % or matin1 Q-1 Not sure of the difference between th both
% With \usepackage[<encoding>]{inputenc} you can directly input accented and other characters. What's important is that <encoding> matches the encoding with which the file has been written and this depends on the operating system and the settings of the text editor.(wish setting ?)
% Q-2: Is inputenc correctec withMac Osx/Texmaker ?
% Codé en UTF-8,permet de reconnaître les lettres accentuées. Le nom à donner entre crochets dépend du système sur lequel je travaille

\usepackage[polutonikogreek, french]{babel} % ou option francais et grec ancien. I'll have a few quotation in grec.
\usepackage[babel]{csquotes} %% Permettent d'utiliser les guillemets sans espaces, le package csquotes s'occupe de l'espacement.

\usepackage[T1]{fontenc}	%% Utilise les fontes de LaTeX en incluant les mots accentués.
‰\package test


\usepackage{lmodern}  %% Font change / or \usepackage{mathptmx} or\usepackage{txfonts} for Times

\usepackage[top=30mm, bottom=30mm ,left=30mm , right=30mm , includeheadfoot , bindingoffset=5mm]{geometry}	% a revoir ou bien margin=28mm



%%%%% Heading & Page style %%%%%%%

% This mayout shoud startwith the introduction
\usepackage{fancyhdr}
\pagestyle{fancy}

%% Heading 
\fancyhead[LE]{\textsc{{leftmark}}% champ gauche, page paire /Leftmark Nom du chapitre pour la classe book
\fancyhead[RO]{rightmark} % champs droit, pages impaire / Rightmark nom section pour classe book.
\renewcommand{headrulewidth}{1pt}

%% Footpage %%%%
\fancyfoot[C]{\textbf{\thepage}}
\renewcommand{\footrulewidth}{0pt} % supprimela ligne horizontale

%%%% Footnote Customize %%%%%
% Remove rench footnote style included with {Babel} {French}
\frenchsetup{FrenchFootnotes=false ,AutoSpaceFootnote=false}


%%%%% Interligne %%%%
\usepckage{setspace}
\onehalfspacing    %% inside the body these{onehalfspace} or {spacing}{n} or {singlespace}


%%%% Bibliography ref included in Tableofcontens %%%%%%%

\usepackage[nottoc, notlof, notlot]{tocbibind}

%%%% Index %%%%%%

\usepackage{makeidx}
\makeindex

%%%% Glossary %%%%%
\usepackage{glossary}   % Verifed if I have this package 
\makeglocary 

%%% Figures&Pictures %%%%%

\usepackage{graphicx}  

%%% Grec - Shortcut %%%%%%

\newcommand*{\textgreek}[1]
\nexcommand*{\tg}[1]{\textgreek{#1}}
% In the body the shortcut will be: \tg{Ulysse lettre latine correspondant aux lettre greques voir ENS...}

%%% Temporary %%%%%%

\usepackage{lipsum} %%%% Used with test version, Will be omitted later on

%%% Introduction %%%%
%This commad is to make sure that Introduction is included in the table of contentes -even so it is not numbered *
%\addcontentsline{toc}{niveau}{\protect\numberline{}titre}
\addcontensline{toc}{chapter}{\protect\numberline}{}introduction}

%%%% Constunized Table of contence %%%%%%
%Part Capital bold letters
\renewcommand{\thechapter}{\Roman{chapter}} % Bold + capital letter
\renewcommand{\thesection}{\arabic{section}} %Small capital letter schape +normal
\renewcommand{\thesubsection}{\arabic{section}.\arabic{subsection}} % only italic letter 
%utiliser le package titlesec pour changer la fonte utilisée dans table of contente



%%%%%%%%%%%%%%%%%%%%%%%%%%%%%%%%%%%%%%%%%%%%%%%%%%%%%%%%%%%%%%%%%%%%%%%%%%%%%%%%%%%%%%%%%
%%%%%%%%%%%%%                  CORPS DE DOCUMENT                        %%%%%%%%%%%%%%%%%
%%%%%%%%%%%%%%%%%%%%%%%%%%%%%%%%%%%%%%%%%%%%%%%%%%%%%%%%%%%%%%%%%%%%%%%%%%%%%%%%%%%%%%%%%

\begin{document}

%%%%%%%%%%%%
\frontmatter %%%%%%%%%%%%%%%%%%%%%%%%%%%%%%%%%%%%%%%%%%%%%%%%%%%%%%%%%%%%%%%%%%%%%%%%%%%
%%%%%%%%%%%%
\include{Frontpage}
\include{Researche_laboratory_presentation}
\include{acknowledgments}
\include{Résumé}    %% plus keywords
\include{Abstract}
\include{dedications}

\tableofcontents
\listoffigures

%%%%%%%%%%%
\mainmatter %%%%%%%%%%%%%%%%%%%%%%%%%%%%%%%%%%%%%%%%%%%%%%%%%%%%%%%%%%%%%%%%%%%%%%%%%%%%
%%%%%%%%%%%
\clearpage

\chapter*{introduction}
\lipsum[2]

\part{Title_1}
\chapter{premier}
\section{section de premier}
\lipsum[3]
\chapter{second}
\section{section seconde}
\lipsum[4]
\lipsum[5]
\part{Title_2}
\chapter*{Conclusion generale}
\lipsum[6]
%%%%%%%%%%%%%%%%%%%%%%%%%%%%%%%%%%%%%%%%%%%%%%%%%%%%%%%%%%%%%%%%%%%%%%%%%%%%
\clearpage
\bibliographystyle{plain-fr} %% option fr les noms sont écritsen petite capitale
\bibliography{name_files}  %% name files withou the extension .bib


%%%%%%%%%%%%%%%%%%%%%%%%%%%%%%%%%%%%%%%%%%%%%%%%%%%%%%%%%%%%%%%%%%%%%%%%%%%%

\printindex

%%%%%%%%%%%
\backmatter %%%%%%%%%%%%%%%%%%%%%%%%%%%%%%%%%%%%%%%%%%%%%%%%%%%%%%%%%%%%%%%%
%%%%%%%%%%%

\chapter{Title_Annexe_1}
\input{file_name}
\chapter{Title_Annex_2}
\input{file_name}

\loadglentries{glossaire-these}



\end{document}


%%%%%%%%%%%%%%%%%%%%%%%%%%%%%%%%%%%%%%%%%%%%%%%%%%%%%%%%%%%%%%%%%%%%%%%%%%%%%%%%%%%%%%%%%%%%%%
%%%%%%%%%%%%%%%%%%%%%%%%%%%%%%%%%%%%%%%%%%%%%%%%%%%%%%%%%%%%%%%%%%%%%%%%%%%%%%%%%%%%%%%%%%%%%%


%%%%%%%%%%  MINIPAGE   %%%%%%%%%%%%%%

\begin{center}\begin{minipage}{10cm}   % Texte length 
Blablababalabalblablabblabablalablablablablablablablablablablabablablablablablablablablablablablablablbbabalba
\end{minipage}
% To include a footnote in a Minipage used the commande \footnotemark{eg Freud S.}. instead of \footnote{eg Freud S.}.
% Mettre la ponctuation après la note de bas de page
% Q1- How to change the size of the lettre, if it has to be a smaller than the rest..   \small in a minipage ?

%%%%%%%%%%  FOOTNOTE  %%%%%%%%%%%%%%

\footnote{blablabla} % Commande

%%%%%%%%%% Figures  %%%%%%%%%%%%%%%%%

\begin{figure}[!h]   % top
\centrering
includegraphics{Pict01.jpg/FileName}
\caption{Pict01, MaïMaï Gédéon}
\label{Pict01}
\end{figure}

% edit figures h(here) t(top) b(botom) p (separat page)

%%%%%%%%%%% Quotation Marks %%%%%%%%%%%

\MakeAutoQuote{“}{“}			% Commande comment l'utiliser ? Mettre les guillemets français entre acolades

%%%%%%%%%% Index  %%%%%%%%%%%%%

% Pour obtenir une entrée dans l'index, il faut l'indiquer par la commande \index{mon_entree}.
% Si accents: on utilise pour cela une arobase (@) et le code \index{sans_accents@avec_accents}

%%%%%%%%% Appendix %%%%%%%%%%%
% https://tex.stackexchange.com/questions/49643/making-appendix-for-thesis

%%%%%%%% Glossary %%%%%%%%%%%

%\usepackage{glossaries}
%\makeglossaries
%\loadglentries{glossaire-these}






















